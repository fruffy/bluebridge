\begin{abstract}
Variety of applications require large scale data processing~\cite{Chen2016}. Many
big data processing frameworks attempt to abstract away the
complexities inherent with distributed data processing by enforcing programming
models~\cite{Dean2004,Zaharia2012,Murray2013}. This leads to the
frameworks imposing undo overhead on the applications and performing worse than
a highly optimized single threaded laptop implementation~\cite{189908}. 

We propose taking the simplicity of a single machine implementation and
combining it with the advantages of a distributed implementation. We envision a
system which turns a datacenter rack or equivalent cluster into a large NUMA
machine. We realize the first part of this system in Camelot, a network
managed distributed shared memory (DSM) backend. 

Camelot is the next generation of a previous system, BlueBridge, which
implemented the basis for network managed DSM. We expand upon BlueBridge to
bring it closer to our envisioned system by adding support for multi-threading,
improved paging policies, and fault tolerance. We evaluate Camelot on these
three axises to determine the performance cost of the added improvements.
\end{abstract}