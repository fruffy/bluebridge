%%%%%%%%%%%%%%%%%%%%%%%%%%%%%%%%%%%%%%%%%%%%%
\section{Implementation}
\label{sec:implementation}
%%%%%%%%%%%%%%%%%%%%%%%%%%%%%%%%%%%%%%%%%%%%%

\dinv is implemented in 8k \footnote{All LOC counts in the paper were
  collected using cloc~\cite{cloc}; test code is omitted from the
  counts.} lines of Go code. It has been tested on Ubuntu 14.04 \&
16.04, and relies on Go 1.6.
%
%% To maintain vector clocks we developed
%% our own vector clock library.  The library 
Dinv implements an optimized version of the vector clock algorithm and
uses a manually-constructed database of functions in \emph{net} and
wrappers for these functions.
%% , and
%% includes a static analysis tool for automatically injecting vector
%% clocks into Go programs. We constructed a  Functions are first-class
%% objects in Go, which allows them to be passed into other functions as
%% arguments.  
We use Go's AST library to build, traverse, and mutate the AST of a
program (e.g., to search for networking functions in our
database).
%%  When a match is detected, we perform a rotation on the
%% AST node containing the networking call and inject our code.
%% 
%% We developed a separate component for instrumenting logging
%% annotations
Dinv's state instrumentation builds on control and data flow
algorithms in GoDoctor~\cite{godoctor}. %% Furthermore we developed a
%% component for merging logs, and a runtime environment which writes
%% variables to a log.  
\dinv requires a working installation of Daikon version 5.2.4 or newer.

%, and takes $0.01s$ to build. 
%
%% Dinv also has a number of dependencies, most notably the
%% Go vector clock library GoVec. Currently DInv requires that user
%% systems incorporate the GoVec library API in their communication layer
%% protocols. The complete list of dependencies can be found in the
%% online installation instructions at
